\documentclass{acm_proc_article-sp}
\usepackage{cite}
\usepackage{graphicx}
\usepackage{url}
\usepackage{array}
\usepackage{multirow}
\usepackage{color}
\usepackage{booktabs}
\usepackage{float}
\usepackage{lipsum}

\IfFileExists{zi4.sty}{\usepackage{zi4}}{\usepackage{inconsolata}}
\usepackage{listings}
\definecolor{bluekeywords}{rgb}{0.13,0.13,1}
\definecolor{greencomments}{rgb}{0,0.5,0}
\definecolor{redstrings}{rgb}{0.9,0,0}
\lstset{language=Java,
	xleftmargin=7mm,
	xrightmargin=2mm,
	frame=single,
	framesep=3pt,
	aboveskip=2em,
	belowskip=1em,
	numbers=left,
	numbersep=9pt,
	captionpos=b,
	tabsize=2,
	keepspaces=true,
	showspaces=false,
	showtabs=false,
	breaklines=true,
	showstringspaces=false,
	breakatwhitespace=true,
	commentstyle=\color{greencomments},
	keywordstyle=\color{bluekeywords},
	stringstyle=\color{redstrings},
	basicstyle=\ttfamily\scriptsize,
	otherkeywords={
		rep,norep,owner,world,any,peer,
		where,reads,writes,pure,impure,
		intersects,disjoint,unique},
	escapeinside={/*@}{@*/}
}

\begin{document}

\title{Ownership Types for Local Reasoning}
\subtitle{CS5218: AY2014/2015 Semester 2, Final Project}

\numberofauthors{3} 

\author{
\alignauthor
Darius Foo\\
		\affaddr{A0097282@u.nus.edu}
\alignauthor
Daryl Seah\\
		\affaddr{A0026468@u.nus.edu}
\alignauthor
Joel Low\\
		\affaddr{A0097630@u.nus.edu}
}



\maketitle
\begin{abstract}
The use of ownership types improves local reasoning, allowing both programmers 
and program analysis tools to reason about the correctness and behaviour of 
programs. This would improve the scalability of program analysis tools. 
However, owing to the difficulty of annotating programs, it is likely that 
ownership types will only be used for safety-critical programs.
\end{abstract}

\section{Introduction}
\label{sec:intro}

% Problem, motivation, and Breadth goes here
The advent of Object-Oriented Programming has allowed programmers to write and 
maintain larger components and programs, owing to the modularisation brought 
about by encapsulation. This increased modularisation of programs ought to 
result in a corresponding ease of comprehension for programmers, since 
programmers are now able to reason about programs at a modular level. 
Likewise, it should follow that static program analysis tools be able to 
improve the speed at which analyses can be completed, especially when analysing 
a large program.

\subsection{Limits of Traditional Mechanisms}
In practice, many escape mechanisms which break the encapsulation of an object 
are used. These mechanisms, while allowing expressiveness in programs, prevent 
the programmer from reaping the benefits of being able to isolate a module and 
reason about its behaviour and effects. We present an example in 
Listing~\ref{code:modular_reasoning_car_engine_1}, originally 
described in \cite{clarke98ownership}.

\begin{lstlisting}[
	float, floatplacement=t,
	caption={Car},
	label=code:modular_reasoning_car_engine_1
]
class Person {}

class Engine {
	void start() { /* ... */}
}

class Car {
	Engine engine; Person driver;

	void start() {
		if (driver != null) {
			engine.start();
		}
	}
}

class Main {
	static void main() {
		Car car = new Car();
		car.start();
		car.engine.start();/*@\label{code:modular_reasoning_car_engine_1_engine_start}@*/
	}
}
\end{lstlisting}

Here, we present a car, with a driver and an engine. The car defines a 
\lstinline|start| method, which will start the engine; the engine can 
only be started if a driver is present. However, on 
line~\ref{code:modular_reasoning_car_engine_1_engine_start} the car's engine 
was started, engine bypassing that check. This should not have been allowed 
because it has broken the encapsulation discipline.

The traditional approach to preventing this is to utilise access specifiers; 
using Java's syntax, there would be four levels of access: \lstinline|public|, 
\lstinline|protected|, \lstinline|private|, and default (package-level) access. 
These access specifiers are annotations on types and variables, limiting 
\emph{visibility} of the identifiers of said types and variables to the 
appropriate scope. Listing~\ref{code:modular_reasoning_car_engine_2}, modified 
from Listing~\ref{code:modular_reasoning_car_engine_1}, demonstrates that 
enforcing encapsulation using access specifiers alone are insufficient.

\begin{lstlisting}[
	float, floatplacement=t,
	caption={Car with Access Specifiers},
	label=code:modular_reasoning_car_engine_2
]
class Person {}

class Engine {
	public void start() { /* ... */ }
}

class Car {
	private Engine engine; private Person driver;
	
	public void start() {
		if (driver != null) {
			engine.start();
		}
	}

	public Engine getEngine() { return engine; }
}

class Main {
	public static void main() {
		Car car = new Car();
		car.start();
		car.getEngine().start();/*@\label{code:modular_reasoning_car_engine_2_engine_start}@*/
	}
}

\end{lstlisting}

In Listing~\ref{code:modular_reasoning_car_engine_2} we annotate each method 
and member variable with an appropriate access specifier. We also define an 
additional \emph{accessor} method, for the situation where a property of the 
car's engine must be accessed outside of the car (an instrument attached to the 
car's electronics to measure performance, for example). However, as seen on 
line~\ref{code:modular_reasoning_car_engine_2_engine_start} the original 
problem has not been resolved: the engine can still be started externally.

While access specifiers can be modified to support such a 
scenario\footnote{Read-only and mutable interfaces can be extracted from the 
class, and the appropriate interface returned from methods. This is not common 
practice; however, the Cocoa runtime for Objective-C is notable for doing 
this.}, the resulting API is clumsy because every time a type is defined, 
programmers must manually define the behaviours allowed by mutable and 
immutable references to the object. There would consequently be an explosion of 
types, making reasoning about the program more difficult.

\section{Ownership Types}
\label{sec:ownership}

% terminology and the basics

\subsection{Topology}
\label{subsec:topo}

\lipsum[3]

\subsection{Encapsulation}
\label{subsec:encap}

\lipsum[4]

\subsection{Applications}
\label{subsec:apps}

\lipsum[5]




\section{Variants}
\label{sec:variants}

\lipsum[6]

\subsection{Clarke's Ownership Types}
\label{subsec:clarke}

\lipsum[7]

\subsection{Boyapati's SafeJava}
\label{subsec:boyapati}

\lipsum[8]

\subsection{Dietl and M\"{u}ller's Universe Types}
\label{subsec:dietl}

\lipsum[9]

\subsection{Cameron's Multiple Ownership}
\label{subsec:cameron}

\lipsum[10]



\section{Evaluation}
\label{sec:eval}

\lipsum[11]




\section{Conclusion}
\label{sec:conclude}

\lipsum[12]








\section{Testing}
\label{sec:test}

% Code Snippets
\begin{lstlisting}[
	float, floatplacement=t,
	caption={Car Engine Example},
	label=code:car_eng_test
]
class Person {}
class Engine<any engineOwner> {
  void start() { /* ... */ }
}

class Car<carOwner, driverOwner> {
  Engine<this> engine; Person<a & ?> driver;
    
  void start() {
    if (driver != null) getEngine().start();
  }
  Engine<world> getEngine() { return engine; }
}
\end{lstlisting}

Reference to Listing~\ref{code:car_eng_test} and Section~\ref{sec:test}.

% Citations
aldrich04domains~\cite{aldrich04domains},\newline
aldrich02aliasjava~\cite{aldrich02aliasjava},\newline
almeida97balloons~\cite{almeida97balloons},\newline
boyapati04safejava~\cite{boyapati04safejava},\newline
boyapati02races~\cite{boyapati02races},\newline
boyapati03innerclass~\cite{boyapati03innerclass},\newline
boyapati03rtsj~\cite{boyapati03rtsj},\newline
cameron07mojo~\cite{cameron07mojo},\newline
cameron10encoding~\cite{cameron10encoding},\newline
clarke03ownership~\cite{clarke03ownership},\newline
clarke02ownership~\cite{clarke02ownership},\newline
clarke13aliasing~\cite{clarke13aliasing},\newline
clarke98ownership~\cite{clarke98ownership},\newline
cunningham08ut~\cite{cunningham08ut},\newline
dietl09gut~\cite{dietl09gut},\newline
dietl11gut~\cite{dietl11gut},\newline
dietl07gut~\cite{dietl07gut},\newline
dietl13ownership~\cite{dietl13ownership},\newline
dietl08dependent~\cite{dietl08dependent},\newline
dietl05jml~\cite{dietl05jml},\newline
hogg91islands~\cite{hogg91islands},\newline
muller02modular~\cite{muller02modular},\newline
muller99universes~\cite{muller99universes},\newline
noble98alias~\cite{noble98alias}



\bibliographystyle{abbrv}
\bibliography{report}

\end{document}
